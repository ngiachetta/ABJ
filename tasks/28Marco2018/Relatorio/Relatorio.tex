\documentclass[12pt]{article}

\usepackage{ifxetex,ifluatex}

\ifnum 0\ifxetex 1\fi\ifluatex 1\fi=0 % if pdftex
  \usepackage[T1]{fontenc}
  \usepackage[utf8]{inputenc}
\else % if luatex or xelatex
  \ifxetex
    \usepackage{mathspec}
    \usepackage{xltxtra,xunicode}
  \else
    \usepackage{fontspec}
  \fi
  \defaultfontfeatures{Mapping=tex-text,Scale=MatchLowercase}
  \newcommand{\euro}{€}
\fi

\usepackage{amssymb,amsmath}
\usepackage{float}
\usepackage{fixltx2e} % provides \textsubscript
\usepackage{fancyhdr}
\usepackage[utf8]{inputenc}
\usepackage[T1]{fontenc}
\usepackage[portuguese]{babel}
\usepackage{graphicx}
\usepackage{indentfirst}
\usepackage{lipsum}
\usepackage{longtable}
\usepackage{pdfpages}
\usepackage[colorlinks=true,urlcolor=blue,citecolor=blue,linkcolor=blue]{hyperref}
\renewcommand{\familydefault}{\sfdefault}
\renewcommand{\baselinestretch}{1.4}

% Paper style
\usepackage[
  paperwidth=21cm,paperheight=29.7cm, % larger paper
  layoutwidth=21cm,layoutheight=27.82cm, % A4 landscape
  layouthoffset=0cm,layoutvoffset=1cm, % 1cm to crop on all sides
  left=2cm,right=2cm,top=2cm,bottom=2cm,
  headsep=\dimexpr3cm-72pt\relax,
  headheight=72pt]{geometry}

\pagestyle{fancy}
\lhead{}
\chead{}
\rhead{\includegraphics[scale = 1]{logo-abj.png}}
\lfoot{}
\cfoot{Associação Brasileira de Jurimetria \\ Rua Gomes de Carvalho, 1356, 2º andar. CEP 04547-005 - São Paulo, SP,
Brasil \\ \url{http://abj.org.br}}
\rfoot{\text{} \\ \text{} \\ \thepage}
\renewcommand{\headrulewidth}{0.4pt}
\renewcommand{\footrulewidth}{0.4pt}


% Header information
\title{Coleta de dados}
\author{}

\providecommand{\tightlist}{%
  \setlength{\itemsep}{0pt}\setlength{\parskip}{0pt}}

\begin{document}

\maketitle

\thispagestyle{fancy}

{
\hypersetup{linkcolor=black}
\setcounter{tocdepth}{2}
\tableofcontents
}
\pagebreak

\section{Base de dados com informações sobre a fase jurídica da
persecução penal à
corrupção}\label{base-de-dados-com-informacoes-sobre-a-fase-juridica-da-persecucao-penal-a-corrupcao}

Como descrito no projeto inicial de pesquisa, o presente estudo contará
com uma lista de processos judiciais, obtidos diretamente dos tribunais
e através da leitura automática de DJEs. Este projeto possue como
recorte a obtenção de dados em quatro estados, sendo eles, São Paulo,
Alagoas, Rio de Janeiro e Distrito Federal.

No que diz respeito a obtenção diretamente dos tribunais, estruturou-se
uma estratégia para coletar informações dos Tribunais de Justiça, das
Justiças Federais e dos Tribunais Regionais Federais dos respectivos
estados citados. Além disso, a coleta teve foco nos processos
categorizados dentro da classe \textbf{inquérito policial} e nas suas
respectivas Jurisprudências e Consultas Processuais, permitindo então
uma consistência maior para as futuras análises.

Cabe destacar que a coleta manual não seria tão ágil devido a quantidade
de processos e consequentemente o tempo atribuido a sua coleta, sendo
assim, automatizamos o processo por meio de uma técnica conhecida como
\emph{``web scraping''}, que compreende a coleta automatizada, também
chamada de raspagem, de informações disponíveis em arquivos HTML das
quais todos os tribunais possuem, uma vez que o site de acesso para eles
são estruturados neste tipo de arquivo.

Porém, apesar de estruturados em HTML, o sistema para coleta de
informações dos processos é diferente entre tribunais e Estados, o que
dificultou a estratégia de coleta. A solução, portanto, foi construir
``rôbos'' especificos para cada site para posteriormente obter as
informações sobre os processos.

Após a coleta dos dados, era preciso estruturá-los em tabelas visando
uma análise consistente das informações. Devido ao fato dos sites terem
formatos diversos, a transformações dos dados brutos em tabelas também
teve que passar por diferentes tipos de estruturação.

A obtenção da lista de processos judiciais através da leitura automática
de DJEs, assim como argumentada no projeto inicial é mais difícil, uma
vez que muitos destes documentos são obtidos no formato PDF. Logo, para
obter estas informações foi necessário contruir ``rôbos'' para fazer a
raspagem dos sites HTML dos diários, estas raspagens então são capazes
de obter o download dos diários que posteriormente irão passar por uma
estruturação de informações, por meio do que chamamos de \emph{mineração
de texto}.

Nas seções seguintes apresentaremos mais detalhadamente sobre como foram
realizadas as coletas, bem como os sites visitados para obtenção das
bases. As seções serão divididas pelo tipo de coleta, isto é, a listagem
dos processos através dos tribunais e pelos DJEs.

\subsection{Listagem dos processos
(Tribunais)}\label{listagem-dos-processos-tribunais}

A tabela abaixo apresenta quais tribunais foram foco de coleta e o
andamento delas até então. A coluna \texttt{Tipo\ do\ tribunal}
apresenta o tribunal foco da coleta, a \texttt{Tipo\ do\ scraper} é para
indicar qual foi a coleta realizada, a \texttt{Implementado} é para
sinalizar se o algoritimo para obter os dados foi realizado e
estruturado.

\begin{table}

\caption{\label{tab:unnamed-chunk-2}Relação de tribunais coletados}
\centering
\begin{tabular}[t]{l|l|l}
\hline
Tipo do tribunal & Tipo do scraper & Implementado\\
\hline
JFAL & Consulta Processual & NÃO\\
\hline
JFDF & Consulta Processual & SIM\\
\hline
JFRJ & Consulta Processual & PARCIAL\\
\hline
JFSP & Consulta Processual & SIM\\
\hline
TJAL & Jurisprudência & SIM\\
\hline
TJAL & Consulta Processual & SIM\\
\hline
TJDFT & Jurisprudência & SIM\\
\hline
TJDFT & Consulta Processual & SIM\\
\hline
TJRJ & Jurisprudência & SIM\\
\hline
TJRJ & Consulta Processual & SIM\\
\hline
TJSP & Jurisprudência & SIM\\
\hline
TJSP & Consulta Processual & SIM\\
\hline
TRF-1 & Jurisprudência & SIM\\
\hline
TRF-1 & Consulta Processual & SIM\\
\hline
TRF-2 & Jurisprudência & SIM\\
\hline
TRF-2 & Consulta Processual & SIM\\
\hline
TRF-3 & Jurisprudência & SIM\\
\hline
TRF-3 & Consulta Processual & PARCIAL\\
\hline
TRF-5 & Jurisprudência & SIM\\
\hline
TRF-5 & Consulta Processual & SIM\\
\hline
\end{tabular}
\end{table}

A coleta foi realizada nos sites oficiais dos tribunais\footnote{Os
  links utilizados para acessar os sites estão no Anexo.}. Sites como o
de São Paulo e Alagoas, por possuirem sistemas semelhantes (e-SAJ),
tiveram rotina parecida de coleta. Além disso, para o caso de São Paulo
especificamente, foi utilizado o pacote \texttt{esaj} desenvolvido pela
ABJ para uso na linguagem R.

Porém, é preciso destacar alguns problemas que dificultaram a coleta,
como por exemplo, \texttt{CAPTCHAs} presentes na consulta processual do
TRF-3 e da JFAL e problemas na busca por inquérito policial na
jurisprudência do TJRJ. Mais especificamente sobre o TJRJ, o site
impossibilitava a busca pela classe de inquérito policial e as busca
apenas por inquérito retornavam resultados não satisfatórios. Sendo
assim, a estratégia utilizada foi realizar a coleta por intervalos
menores de tempo e filtrar os casos desejados posteriomente. Apesar dos
problemas relatados, cerca de 85\% dos códigos para os respectivos
tribunais estão implementados de forma integral, isto é, os códigos de
coleta e estruturação dos dados estão feitos. Aqueles sinalizados como
``PARCIAL'' significam que possuem o código de estruturação feito, porém
o de obtenção está em andamento.

\subsection{Listagem dos processos
(DJE)}\label{listagem-dos-processos-dje}

A coleta de dados dos Diários de Justiça Estaduais segue os passos da
listagem anterior, isto é, é preciso utilizar a técnica de \emph{web
scraping} para coletar informações sobre os diários, que normalmente são
disponibilizados em PDF. Após a coleta destas informações e download dos
arquivos, é preciso também transformá-las em arquivos que permitem uma
análise consistente.

A tarefa relacionada com a coleta dos dados está na sua etapa final, uma
vez que a maioria dos diários já possui algoritmos para obter os
arquivos em PDF e estruturá-los em arquivos de texto. A estratégia
utilizada para a coleta e estruturação dos dados compreende baixar todos
os diários de interesse, neste caso relacionados com a área Judicial,
transformá-los em arquivos de texto para que em seguida possamos
utilizar expressões regulares\footnote{Expressões regulares são
  sequências específicas utilizadas para encontrar padrões em
  caracteres.} de acordo com assuntos específicos, obtendo as pagínas
que posteriormente coletaremos os números CNJ.

Com o número CNJ em mãos podemos executar os códigos das Consultas
Processuais e assim estruturar as respectivas bases de dados.

\section{Base de dados de perfis dos
entrevistados}\label{base-de-dados-de-perfis-dos-entrevistados}

Paralelamente a coleta de dados dos fluxos dos processos penais dos
casos de corrupção nos estados previamente selecionados, a pesquisa se
propõe a criar um banco de perfis dos agentes chave desse processo, a
saber: juízes federais e estaduais, procuradores federais e estaduais e
delegados federais e estaduais.

Para além de variáveis demográficas, como idade, sexo, ocupação,
escolaridade, local de nascimento etc., o banco visa combinar
características referentes a trajetória de carreira dos entrevistados,
se passaram pelo setor privado ou não, se possuem formação secundária,
se possuem algum tipo de vinculação a associações ou movimentos sociais.
Assim, pode-se cruzar essas variáveis, criando grupos de perfis
(clusters) e analisar as relações entre esses grupos e os resultados das
decisões dos processos.

\subsection{Identificação}\label{identificacao}

A obtenção desse banco de perfis se deu, em primeiro lugar, pela
identificação dos atores pertinentes; através dos sites oficiais dos
Tribunais de Justiça dos quatro estados selecionados e dos sites dos
Tribunais Regionais de Justiça, foi possível obter acesso às listas de
juízes, que foram filtradas pelas categorias ``criminal'', ``crim'',
``vara única'' ou ``vara cumulativa'', ou no próprio site ou em
comparação com o banco de dados interativo da Conselho Nacional de
Justiça (nesse caso, foi produzida uma nova tabela e realizado o
cruzamento entres as comarcas e os nomes). As identificações dos
procuradores se deram por meio dos sites da Advocacia Geral da União e
do Ministério Público em cada estado, sendo as listas dos procuradores
estaduais encontradas nos tópicos ``Colégio de Procuradores Estaduais'',
nos sites do MP, ao passo que para os procuradores federais essa
informação constava nas seções institucionais dos site da AGU,
geralmente no tópico ``Quem é Quem''. Por fim, os delegados foram
identificados nas páginas da Polícia Federal e da Polícia Civil, sendo
possível realizar uma busca por estados a partir da página principal da
PF, no caso dos delegados federais. No entanto, em relação os delegados
estaduais essa informação só foi encontrada na página da Polícia Civil
do Rio de Janeiro.

É importante ressalvar que as listas de identificação não estão todas
completas, isso se deve a disparidade da disponibilização de informações
entre os sites, às vezes entre os sites de uma mesma instituição, como
no caso dos Tribunais de Justiça, que ora apresentam informações
agregadas, incluindo diversos servidores em uma mesma lista, ora
apresentam documentos separados e ora apresentam buscas com filtros na
própria interface do site, ao invés de um documento com uma lista,
dificultando a busca e a extração da informação necessária.

Outra razão para incompletude das listas é a própria falta de informação
nos sites oficiais, como no caso dos sites das Policias Civis do
Alagoas, do Distrito Federal e de São Paulo que sequer apresentam alguma
seção informando quais são os responsáveis pelas delegacias.

A busca pela identificação dos possíveis entrevistados prosseguirá com
telefonemas às corregedorias dos órgãos em que não obtivemos informações
pelos sites ou que as informações estavam incompletas, com o intuito de
cobrir os \emph{missing values} das tabelas com o fornecimento da
informação diretamente dos órgãos.

\subsection{Perfis}\label{perfis}

Em decorrência dos problemas na obtenção das identificações dos
possíveis entrevistados, a fase de levantamento dos perfis dos agentes
encontra-se em defasagem até o presente momento.

Algumas tentativas e esforços já foram realizados nesse sentido
considerando as identificações que já possuímos; ao se procurar
individualmente os nomes de delgados, procuradores e juízes em uma busca
simples na internet, os resultados foram variados, podendo aparecer
informações em plataformas profissionais como Lattes, LinkedIn,
Escavador etc. ou até mesmo mais pessoais como Facebook, por exemplo. No
entanto, esses experimentos mostraram a inconstância que essas
informações podem ser obtidas, como se tratam de plataformas em que os
usuários se cadastram voluntariamente e colocam as informações que
desejam, era comum não encontrar o entrevistado cadastrado em nenhuma
plataforma ou em apenas alguma delas, não havendo uma base comum em que
se possa extrair informações para todos eles, dificultando comparações e
aproximações.

Outra dificuldade encontrada foi a ausência de dados biográficos dos
investigados tanto nas plataformas oficiais, via de regra, quanto em
buscas livres na internet, algo que era esperado, por se tratarem de
funcionários públicos concursados ou de carreira.

Na tentativa de automatizarmos as buscas, foram testados dois pacotes do
software R com o intuito de criar loops de procura e raspagem de dados
nas plataformas Lattes e Facebook, porém não obtivemos sucesso. Uma
provável dificuldade nessa tentativa talvez seja a de lidar com o léxico
da procura, uma vez que resultados idênticos ou semelhantes aos nomes
dos procurados pode atrapalhar a busca e conduzir a informações
desviantes.

Por fim, novas tentativas serão feitas nesse sentido e considera-se caso
essa opção de busca on-line não mostrar-se viável, anexar perguntas de
perfil aos questionários enviados aos entrevistados.

\section{Anexo}\label{anexo}

\subsection{Lista de sites
consultados}\label{lista-de-sites-consultados}

\paragraph{Jurisprudência}\label{jurisprudencia}

\begin{itemize}
\item
  TJSP: \url{https://esaj.tjsp.jus.br/cjsg/consultaCompleta.do}
\item
  TJAL: \url{https://www2.tjal.jus.br/cjsg/resultadoCompleta.do} e
  \url{http://www.jurisprudencia.tjal.jus.br/Default.aspx}
\item
  TJRJ:
  \url{http://www.tjrj.jus.br/search?site=juris\&client=juris\&output=xml_no_dtd\&proxystylesheet=juris}
\item
  TJDFT:
  \url{https://pesquisajuris.tjdft.jus.br/IndexadorAcordaos-web/sistj?visaoId=tjdf.sistj.acordaoeletronico.buscaindexada.apresentacao.VisaoBuscaAcordao}
\item
  TRF-1: \url{https://jurisprudencia.trf1.jus.br/busca/resultado.jsp}
\item
  TRF-2: \url{http://www10.trf2.jus.br/consultas/}
\item
  TRF-3: \url{http://web.trf3.jus.br/base-textual}
\item
  TRF-5: \url{http://www.cjf.jus.br/juris/unificada/Resposta}
\end{itemize}

\paragraph{Consulta Processual}\label{consulta-processual}

\begin{itemize}
\item
  TJSP: \url{http://esaj.tjsp.jus.br/esaj/portal.do?servico=190090}
\item
  TJAL: \url{https://www2.tjal.jus.br/cposg5/search.do}
\item
  TJRJ:
  \url{http://www4.tjrj.jus.br/ejud/WS/ConsultaEjud.asmx/DadosProcesso_1}
\item
  TJDFT:
  \url{https://cache-internet.tjdft.jus.br/cgi-bin/tjcgi1?NXTPGM=plhtml01\&MGWLPN=SERVIDOR1\&submit=ok\&SELECAO=1\&CHAVE}=\&ORIGEM=INTER
\item
  TRF-1:
  \url{https://processual.trf1.jus.br/consultaProcessual/processo.php}
\item
  TRF-2: \url{http://www.trf2.gov.br/cgi-bin/pingres-allen}
\item
  TRF-3:
  \url{http://web.trf3.jus.br/consultas/Internet/ConsultaProcessual}
\item
  TRF-5: \url{http://www.trf5.jus.br/cp/cp.do}
\item
  JFSP: \url{http://csp.jfsp.jus.br/csp/consulta/consinternet.csp}
\item
  JFAL:
  \url{https://jef.jfal.jus.br/cretainternetal/consulta/processo/pesquisar.wsp}
\item
  JFRJ: \url{http://procweb.jfrj.jus.br/portal/consulta/cons_procs.asp}
\item
  JFDF:
  \url{https://processual.trf1.jus.br/consultaProcessual/processo.php}
\end{itemize}

\end{document}
